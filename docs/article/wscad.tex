\documentclass[12pt]{article}

\usepackage{sbc-template}

\usepackage{graphicx,url}

\usepackage[brazil]{babel}
\usepackage[utf8]{inputenc}
\usepackage[T1]{fontenc}

\sloppy

\title{Artigo pro Cuda Fuzzy}

% Edevaldo Braga dos Santos
% Giovane de Oliveira Torres
% Guilherme Pereira Paim
% Renan Zafalon da Silva
% Vitor Alano de Ataides

\author{Edevaldo Braga dos Santos\inst{1}, Giovane de Oliveira Torres\inst{1}, Guilherme Pereira Paim\inst{1},\\ Renan Zafalon da Silva\inst{1}, Vitor Alano de Ataides\inst{1}, Maurício Lima Pilla\inst{1}}

\address{Universidade Federal de Pelotas \\
Pelotas, RS - Brasil\\
  \email{\{edevaldo.santos,gdotorres,gppaim,renan.zafalon,vaataides,pilla\}@inf.ufpel.edu.br}
}

\begin{document}

\maketitle

\begin{abstract}

Abstract aqui.

\end{abstract}

\begin{resumo}

Resumo aqui.

\end{resumo}

\section{Introdução}

% Logica fuzzy: colocar onde? %

\subsection{Lógica fuzzy}

	Existem diversos casos onde classes de objetos não pertencem totalmente a um conjunto. Baseado nisto, Zadeh definiu a teoria dos conjuntos \textit{fuzzy}~\cite{zadeh:65}, o que visa tratar problemas de imprecisão ao classificar dados no mundo real. Os conjuntos \textit{fuzzy} possuem aplicações em sistemas de controle e de suporte à decisão, onde a descrição do problema não é feita de forma precisa~\cite{weber:03}.
	
	Utilizando-se dos conjuntos \textit{fuzzy}, tem-se a base para a lógica \textit{fuzzy}, sendo construído a partir da lógica proposicional. Com isto, os operadores foram definidos à partir dos já estabelecidos na lógica clássica, com a adição de outros para fins práticos~\cite{tanscheit:04}. Uma característica interessante que diferencia a lógica tradicional da \textit{fuzzy} é que na primeira os valores que são utilizados atendem a condição de serem verdadeiros ou falsos (0 ou 1). Já na segunda, trabalha-se com conjuntos \textit{fuzzy} -- estes podem assumir um valor que pertence ao intervalo $[0, 1]$, o que permite que um conjunto \textit{fuzzy} possa ser representado por uma infinidade de valores~\cite{klir:95}.	
	
\bibliographystyle{sbc}
\bibliography{wscad}

\end{document}
